\documentclass[a4paper,12pt]{article}
\usepackage[utf8]{inputenc}
\usepackage[T1]{fontenc}
\usepackage[francais]{babel}
\usepackage{amsmath}

\begin{document}
\begin{center}
\fbox{\LARGE Formules trinôme du second degré}
\end{center}
\bigskip

\[P(x) = ax^2+bx+c = a(x-\alpha)^2+\beta\]

Le sommet de la parabole a pour coordonnées $(\alpha;\beta)$ avec : 
\[\alpha = \dfrac{-b}{2a} \qquad\text{et}\qquad \beta = P(\alpha)\]


Calcul du discriminant : $\boxed{\Delta = b^2-4ac }$
\bigskip

$\boxed{\Delta < 0} \implies$ Pas de racines donc pas de factorisation.

La parabole ne coupe pas l'axe des abscisses et elle est tournée :
\begin{itemize}
  \item Branche vers le haut si $a>0$
  \item Branche vers le bas si $a<0$
\end{itemize}
\bigskip

$\boxed{\Delta = 0} \implies$ 1 racine double $x_0$ tel que 
$P(x) = a(x-x_0)^2$ avec :
\[
x_0=\dfrac{-b}{2a}
\]
La parabole coupe l'axe des abscisses en un point et son allure dépend de $a$.
\bigskip

$\boxed{\Delta > 0} \implies$ 2 racines distinctes tel que $ P(x) = a(x-x_1)(x-x_2)$ avec :
\[
x_1=\dfrac{-b-\sqrt{\Delta}}{2a} \qquad\text{et}\qquad x_2=\dfrac{-b+\sqrt{\Delta}}{2a}
\]
La parabole coupe l'axe des abscisses en deux points et son allure dépend de $a$.
\bigskip

\textit{Remarque : Pour le tableau de signes, imaginer l'allure de la parabole et observer sa position par rapport à l'axe des abscisses ou utiliser les règles sur le signe de $a$.}
\end{document}
